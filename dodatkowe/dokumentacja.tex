\documentclass[12pt]{article}
\usepackage[utf8]{inputenc}
\usepackage{amsmath}
\usepackage[plmath,OT4]{polski}

\begin {titlepage}
\title{SuperShop. Dokumentacja}

\date{\today}
\author{Kamil Bałdyga}

\end{titlepage}

\begin{document}
  \maketitle \newpage

\section{Instalacja}
  \subsection{Wymagania}
	Do poprawnego działania programu wymagana jest
	\begin {itemize}
		\item wirtualna maszyna erlanga (projekt pisany oraz testowany był w wersji: Erlang R13B04 (erts-5.7.5))
		\item PostgeSQL (pisany oraz testowany na wersji 8.4.3)
	\end{itemize}
  \subsection{Uruchomienie}
	W celu uruchomienia serwera hostującego aplikację, należy wykonać skrypt znajdujący się w katalogu Quickshart (quickstart.sh). Standardowy port nasłuchujący (z krótym należy się łączyć) to 8000 (można to zmienić w pliku Quickstart/src/quickstart\_app linia 4)
  \subsection{Konfiguracja}
	\begin{itemize}
		\item Wymagane jest uprzednie skonfigurowanie połączenia z bazą danych. Należy wygenerować odpowiedni connection string (adres serwera bazy danych, login, ...), oraz umieścić go w pliku: Quickstart/src/database.erl w linii 23 (zgodnie z zamieszczonym przykładem).
		\item Należy zadbać o to, aby w bazie znalazł się użytkownik o nazwie "admin" (bez cudzysłowów), ponieważ to on, jako sprzedawca będzie miał dostęp do dodatkowych funkcjonalności. Hasła w bazie danych szyfrowane są metodą md5, nie istnieje więc możliwość przypomnienia hasła. Administrator bazy danych, może jedynie zmienić je na nowe.
	\end{itemize}
 
\section{Podręcznik użytkownika}
  \subsection{Podręcznik administratora}
	Panel administratora znajduje się pod adresem /admin. Tam też można dodawać nowe kategorie, oraz produkty. 
	\begin {itemize}
		\item  Dla produktu można dodać zdjęcie, jednak obsługiwany jest tylko format .jpg (produkt bez obrazka wyświetlał będzie domyśliny obrazek Quickstart/static/images/produkty/no\_image.jpg). Konieczne jest wypełnienie wszystkich pól. Jeśli produkt nie zostanie przypisany do żadnej z kategorii widoczny będzie tylko i wyłącznie pod adresem /product (tam wypisywane są wszystkie produkty).
		\item Dodanie nowej kategorii to wypełnienie tylko 2 pól i kliknięcie przycisku dodaj.
	\end {itemize}
	Ze strony panelu /admin dostać się można również do strony /zlecenia. Tam też administrator może zmieniać stan produktów z nieobsłużonego, na obsłużony. W tabeli znajdują się informacje o identyfikatorze zamowienia, identyfikatorze produktu, typie płatności, identyfikator zamawiającego, oraz nazwa produktu. Po kliknięciu przycisku zlecenie otrzyma status obsłużone i nie będzie więcej wyświetlane (nadal będzie istniało w bazie danych). Identyfikator zamawiającego jest linkiem, który przenosi na stronę wyświetlającą dane użytkownika wymagane do zakończenia transakcji.
  \subsection {Podręcznik użytkownika}
	Każdy klient ma możliwość przeglądania produktów, oraz dodawania ich do koszyka. Produkty przypisane są do kategorii (może być kilka). Na stronie widnieje informacja ile produktów znajduje się w każdej kategorii. Puste kategorie nie są wyświetlane. Każdy produkt opisany jest przez zdjęciue, nazwę, krótki opis, cenę, oraz informację o dostępności. Po kliknięciu nazwy produktu klient przenoszony jest na stronę z dokładnymi informacjami na temat produktu. Tam też znajduje się przycisk dodania do koszyka.
Po wybraniu odpowiednich produktów, należy przejść do swojego koszyka (link po lewej stronie, bądź bezpośrednio /cart). Po kliknięciu przycisku finalizującego zakupy:
	\begin {itemize}
		\item niezalogowany klient proszony jest o zalogowanie, po zalogowaniu wykonuje się akcja 2
		\item poprawnie zalogowany klient składa zamówienie. Informacja o poprawnym złożeniu zamówienia pojawi się na stronie.
	\end {itemize}
	Do zalogowania potrzebne jest konto użytkownika. Aby je utworzyć należy kliknąć "Sign Up!" na stronie login (bezpośrednio /add\_account), oraz uzupełnić nazwę użytkownika, hasło, oraz email. Od tej pory można składać poprawne zamówienia. Uzupełnić, lub zmienić dane kontaktowe można w panelu klienta.


  % This is a comment; it is not shown in the final output.
  % The following shows a little of the typesetting power of LaTeX
%  \begin{align}

%  \end{align}

\end{document}